X-ray imaging is a critical diagnostic tool used by medical professionals to diagnose and treat various illnesses and injuries. While the use of X-rays is essential to provide immediate, life-saving results, the amount of radiation dosage patients receive can be more than medically necessary. This excessive radiation exposure can increase the risk of cancer, DNA damage, and other potential health risks, making it crucial to minimize the amount of radiation exposure patients receive during X-ray imaging. \cite{RadiationEffects} 

While one could minimize radiation dosage by adding shielding and/or changing the position of the target/x-ray source and obtain a reported dosage from the x-ray machine software, this approach has limitations. First, it is time-consuming in often urgent situations, and it still does not guarantee the optimal imaging parameters. Secondly, when it comes to resource-limited settings, the increased cost of imaging procedures due to this optimization procedure may not be financially viable.

To address these challenges, a research project focusing on the development and validation of a computational modeling approach for X-ray attenuation and medical imaging simulation using ray tracing is in progress in Saint Vincent College's Physics Department. This approach will be able to accurately simulate the image one might obtain from an X-ray machine, while also being able to approximate the associated dosage. Accurate computer simulations enable the user to easily add shielding and change the X-ray source orientation, allowing one to optimize the imaging procedure by finding the best quality image that can be obtained with a minimal dosage of ionizing radiation.

In particular, this approach to X-ray simulation uses ray tracing, which is a common graphics technique that simulates particles of light traveling in a 3D environment. It is in this 3D environment that one can place models of objects and trace the path of the light particles through the objects. By using the objects' attenuation coefficient, which is a parameter that describes how strongly a material weakens or reduces the strength of X-rays as they pass through the material, and the distance light travels through the object, one is able to calculate the attenuation of the X-ray and simulate the expected X-ray image.

Currently, this ray-tracing model assumes a single energy x-ray, monoenergetic, is traveling through the objects in the scene; however, with actual X-ray machines, a wide range of energies, polyenergetic, is observed. This observed spectrum has the following explanation: inside an X-ray tube, electrons build up on a filament and are accelerated to an anode as a result of a high voltage being applied across the filament (cathode) and the X-ray target (anode). As the electrons hit the target, often made of tungsten, the electrons have a low probability of being slowed down and deflected by the positive charge of the protons in the target nucleus. The kinetic energy of the electrons lost due to this attraction is converted to an electromagnetic wave of equal energy and increases with proximity to the nucleus.  This phenomenon is called \textit{bremsstrahlung radiation} and is the only source of the energy spectrum at a kVp below 69.5 keV.  In the extremely rare case that an electron loses all of its kinetic energy due to this nucleus interaction, the maximum energy x-ray is produced and is equal to the accelerating voltage of the tube, which is referred to as the spectrum's kVp. Once a tube's kVp rises above 69.5 keV, the binding energy of tungsten's K shell, a vacancy will be left in the K shell and electrons from the L shell, which has a binding energy of 11.5 keV, will fill this vacancy, producing an x-ray of energy $69.5 - 11.5 = 57.0$ keV. As a result, as the kVp of the tube increases, intensity spikes as a result of electron transitions will appear in the X-ray energy spectrum. \cite{Seibert}

Therefore, to better model reality, it would be best to redesign the ray-tracing model to produce photons according to an energy distribution; however, this would involve thousands of photons per pixel, which would drastically increase the render time of the simulation. An alternative option would be to find the singular X-ray energy that would result in the same attenuation as a particular energy spectrum produced by a tube, which is called the effective energy. \cite{Seibert} With this energy, one would be able to obtain approximately the same results as the spectrum of energies produced by the x-ray machine, eliminating the need to complicate the model and increase computation time. While this effective energy technique would solve many of the simulation's problems, the factors that affect the effective energy must first be determined.



