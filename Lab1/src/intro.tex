Background

Radioactivity: 
	The three main kinds of radiation are alpha, beta, and gamma radiation. Alpha radiation is a helium nucleus, two protons and two neutrons. It has a +2 charge, is relatively heavy, and is easily shielded. Beta radiation comes in two varieties, 

Geiger $M\ddot{u}ller$ Tubes:
	In this experiment, we will be using Geiger $M\ddot{u}ller$ tubes to measure radiation. Geiger $M\ddot{u}ller$ (GM) counters are the most widely used tool for radiation detection because of their accuracy and simple design. A GM counter is able to detect alpha, beta, X-ray, and gamma radiation, giving it high versatility. Their internal construction is a cylindrical tube with a rod running down the center. The tube is filled with a gas, usually neon, argon, or helium, that will be ionized by radiation. A potential is applied between the tube and the inner rod, so that when radiation enters the chamber and ionizes the gas, there is a current flow between the tube and the rod. This current flow is detected by further circuitry and marked as a radiation detection event.\cite{Centric} In addition to the GM counter, we will be using a ST360 Radiation Counter to count and track radiation detections.


Radiation Shielding:

Overview
	We will go through a four step process in this lab. First, we will determine the operating voltage of the GM counter, which we will use for the rest of the experiment. Second, we will measure the background radiation to subtract from our later radiation rate measurements. Third, we will calculate the dead time of the GM counter using the two source method. Finally, we will calculate the shielding coefficient of lead and the relationship between the shielding thickness and the radiation attenuation. 

Expected Relationship between shielding thickness and radiation attenuation
Functional Relationship:
Describe parameters you define in terms of their physical effects
Determine units for each parameter
Research that establishes its validity

Centric Geiger Muller Tubes Manual