\par Intro about radiation? 
\par The three main kinds of radiation are alpha, beta, and gamma radiation. Alpha radiation is a helium nucleus, two protons and two neutrons. It has a +2 charge, is relatively heavy, and is easily shielded by paper. Beta radiation comes in two varieties, $\beta^+$ and $\beta^-$. $\beta^+$ radiation occurs when a neutron decays into a proton, an electron, and a neutrino. $\beta^-$ radiation occurs when a proton decays into a neutron, a positron, and a neutrino. Both beta radiations are charged and can be shielded by thin metal sheets. Finally, gamma radiation is high energy photons, in the MeV range and higher. Gamma radiation is electrically neutral and needs heavy shielding, as it can penetrate lead in high enough concentrations \cite{Radiation Source}. In our experiment, we will be using Cobalt-60 and Cesium-137, both of which are gamma sources.

\par In this experiment, we will be using Geiger $M\ddot{u}ller$ tubes to measure radiation. Geiger $M\ddot{u}ller$ (GM) counters are the most widely used tool for radiation detection because of their accuracy and simple design. A GM counter is able to detect alpha, beta, X-ray, and gamma radiation, giving it high versatility. Their internal construction is a cylindrical tube with a rod running down the center. The tube is filled with a gas, usually neon, argon, or helium, that will be ionized by radiation. A potential is applied between the tube and the inner rod, so that when radiation enters the chamber and ionizes the gas, there is a current flow between the tube and the rod. This current flow is detected by further circuitry and marked as a radiation detection event.\cite{Centric} In addition to the GM counter, we will be using a ST360 Radiation Counter to count and track radiation detections. 

\par We will go through a four step process in this lab. First, we will determine the operating voltage of the GM counter, which we will use for the rest of the experiment. Second, we will measure the background radiation to subtract from our later radiation rate measurements. Third, we will calculate the dead time of the GM counter using the two source method. Finally, we will calculate the shielding coefficient of lead with a gamma ray source and the relationship between the shielding thickness and the radiation attenuation.

\begin{comment}
\par We expect the radiation attenuation to follow an exponential decay as a function of lead shielding thickness. The intensity (I) after passing through a lead shield of thickness X is given by the equation
\begin{equation}
I = I_0e^{-\mu X},
\end{equation}
where $I_0$ is the initial intensity and $\mu$ is the attenuation coefficient. To solve this equation for $\mu$, we set the final intensity equal to half the initial intensity, which will occur after a thickness $X_{1/2}$, the half thickness:
\begin{equation}
1/2 I = I_0e^{-\mu X_{1/2}}.
\end{equation}
Solving Equation (2) for $\mu$ gives
\begin{equation}
\mu = ln(2)/{X_{1/2}}.
\end{equation}
We will measure and calculate $\mu$ and $X_{1/2}$ for Cobalt-60 gamma radiation attenuation through lead, expecting an exponential relationship between the shielding thickness and the radiation attenuation\cite{Spectrum}. 
\end{comment}
%Citations for later
%SpecTech Student Manual
%Some radiation source
%Centric Geiger Muller Tubes Manual