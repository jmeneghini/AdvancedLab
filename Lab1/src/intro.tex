Background

Radioactivity: 

Geiger M$\ddot$uller Tubes:
In this experiment, we will be using Geiger Muller tubes to measure radiation. Geiger Muller (GM) counters are the most widely used tool for radiation detection because of their accuracy and simple design.^1 A GM counter is able to detect alpha, beta, X-ray, and gamma radiation, giving it high versatility. Their internal construction is a cylindrical tube with a rod running down the center. The tube is filled with a gas, usually neon, argon, or helium, that will be ionized by radiation. A potential is applied between the tube and the inner rod, so that when radiation enters the chamber and ionizes the gas, there is a current flow between the tube and the rod. This current flow is detected by further circuitry and marked as a radiation detection event. In addition to the GM counter, we will be using a ST360 Radiation Counter to count and track radiation detections.


Radiation Shielding:

Goal
Find the dead time of the detector and the shielding coefficient

Overview
Operating Voltage, Background, Dead time, Radiation Shielding


Expected Relationship between shielding thickness and radiation attenuation
Functional Relationship:
Describe parameters you define in terms of their physical effects
Determine units for each parameter
Research that establishes its validity

1: Centric Geiger Muller Tubes 