
\par \indent To run the experiment, we first had to determine the operating voltage for high sensitivity for the GM tube. This was done by sweeping the operating voltage from 0V - 1200V in 20V increments for 30 seconds on a Spectrum Techniques ST360 GM tube. This data was then plotted to determine a range in which the operating voltage would be sensitive enough without causing dialectric breakdown and damaging the equipment. As seen in figure [], the operating voltage for high sensitivity is within the 800V-1100V range. Thus, 900V was chosen as it was within this range at a convenient point and agreed with the value provided by the manufacturer.

\par Once the operating voltage was determined, the background radiation count was measured to account for noise in the count rate. This was done by running the GM counter for 1000 runs at 1s intervals with no radioactive samples. This background count rate was then subtracted from each data set for count rate measured to minimize noise and gain more accurate results.

\par Another method in reducing the uncertainty for our experiment was calculating the dead time for the GM tube. Dead time is the period of time in which the positive ions take to reach the cathode and the tube becomes insensitive to radiation. Because count rates for radioactive samples are essentially random, we can attempt to correct this random statistical process to determine a true count rate of a substance.  Since the decay of radioactive nuclei can be described by a Poisson distribution, we relate the true count rate as
\begin{equation}
r = Re^{-RT}, 
\label{eq:ActualCountRate}
\end{equation}\cite{Spectrum}
where $r$ is the measured rate, $R$ is the true count rate, and $T$ is the dead time. If we take an approximation of this true count rate with a second-order Taylor Series approximation we see that
\begin{equation}
r \approx R(1-RT).
\label{eq:ApproxCountRate}
\end{equation}
Rearranging Equation (\ref{eq:ApproxCountRate}) for true count rate, we have that
\begin{equation}
R \approx \frac{r}{1-rT}
\label{eq:TrueCountRate}
\end{equation}

\par By accounting for the dead time in our experiment, we can calculate the true count rate of a radioactive sample. This is done by implementing the two-source method. When combining two sources, we measure the combined activity, $r_3$, as 
\begin{equation}
r_1 + r_2 = r_3 + b,
\label{eq:combinedcountrate}
\end{equation} \cite{Spectrum}
where $r_1$ and $r_2$ are the individual counts and $b$ is the background count. If each count rates are corrected for dead time, the previous equation becomes
\begin{equation}
\frac{r_1}{1-r_1 T} + \frac{r_2}{1-r_2 T} = \frac{r_3}{1-r_3 T} + b.
\label{eq:combinedwithdeadtime}
\end{equation}
The background count is not corrected because it is already negligible as is. This means we can transform Equation (\ref{eq:combinedwithdeadtime}) for dead time into a quadratic by
\begin{equation}
r_1 r_2 r_3 T^2 - 2r_1 r_2 T + r_1 + r_2 - r_3 = 0.
\label{eq:quaddeadtime}
\end{equation}
Because $T$ is roughly on the order of microseconds, we can also negate the $T^2$ term and reduces the equation to the final dead time calculation of 
\begin{equation}
T = \frac{r_1 + r_2 - r_3}{2r_1 r_2}.
\label{eq:deadtime}
\end{equation}
To determine dead time through this two-source method in Equation (\ref{eq:deadtime}), we used one Co-60 source and one Cs-137 source and measured both individual count rates. Then both sources were measured together, roughly stacked on top of each other since we did not have half-sources, for the total count rate. Then the dead time was calculated by the above equation.

%Taken from Sam
\par Using the true count rate calculated by the measured dead time, we can use this value to understand how gamma ray radiation interacts with different materials. We expect the radiation attenuation through a material to follow an exponential decay as a function of thickness. The intensity $I$ after passing through a lead shield of thickness $X$ is given by the equation
\begin{equation}
I = I_0e^{-\mu X},
\end{equation}\cite{Knoll}
where $I_0$ is the initial intensity and $\mu$ is the attenuation coefficient. To solve for $\mu$, we set the final intensity equal to half the initial intensity, which will occur after a thickness $X_{1/2}$, the half thickness:
\begin{equation}
1/2 I = I_0e^{-\mu X_{1/2}}.
\label{eq:attenuation}
\end{equation}
Solving Equation (\ref{eq:attenuation}) for $\mu$ gives
\begin{equation}
\mu = ln(2)/{X_{1/2}}.
\end{equation}
Varying different thickness of lead shielding from 1.625mm to 6.350mm and measuring the resulting count rate, we determine the relationship between attenuation and material thickness.

