We have presented the operating voltage and dead time for our Geiger M$\ddot{u}$ller tubes, as well as the shielding coefficient for lead with gamma sources. More specifically, we presented the operating voltage as $900$ Volts and the dead time as $0.0942 \pm 0.02$ seconds. We have presented our results of the attenuation coefficient for Cs-137 as $11.387 \pm 0.8$ mm. This result was obtained by fitting the data on a Count Rate vs. Thickness for Lead-Shielded Cs-137. Looking at the range of radiation energy emitted by our source, between $1.17$ and $1.333$ MeV, we chose to use a theoretical calculation of the attenuation coefficient for lead from the NIST website. The attenuation coefficient of lead from a radiation source which emits $1.25$ MeV of energy is reported as $14.994$ mm. Comparing the result we obtained through data collection and analysis with the reported value from NIST, it shows that these values are not statistically similar with the theoretical value more than one standard deviation larger than the value we found through analysis.
\\ \\
Summarize Key Findings from the research and link them to the initial research question

Place in the findings in context

Mention and discuss any unexpected results

Address limitations or weaknesses in the research

Provide a brief look at potential follow-up research studies

Conclude with a restatement of the most significant findings and their implications
\\ \\
Operating voltage = $900$ V \\
Dead time = $0.0942 \pm 0.02$ seconds
Reported Background Count Rate  = $0.424 \pm 0.02$ counts per second \\ 