\par We have presented the operating voltage and dead time for our GM tubes, as well as the shielding coefficient for lead with gamma sources. More specifically, we presented the operating voltage as $900$ Volts and the dead time as $0.074149 \pm 0.007$ seconds. We have presented our results of the attenuation coefficient for Cs-137 as $11.387 \pm 0.8$ mm. This result was obtained by fitting the data on a Count Rate vs. Thickness for Lead-Shielded Cs-137. The data used in this analysis was obtained through $5$ runs at $300$ seconds of lead-shielded Cs-137 at different thicknesses. Looking at the range of radiation energy emitted by our source, between $1.17$ and $1.333$ MeV, we chose to use a theoretical calculation of the attenuation coefficient for lead from the NIST website \cite{Shielding}. The attenuation coefficient of lead from a radiation source which emits $1.25$ MeV of energy is reported as $14.994$ mm. Comparing the result we obtained through data collection and analysis with the reported value from NIST, it shows that these values are not statistically similar with the theoretical value more than one standard deviation larger than the value we found through analysis. In our analysis of the data for lead-shielded Cs-137 counts, we took into account both the background counts and the dead time of our GM counters. In our analysis of lead-shielded Co-60 count data, we obtained an attenuation coefficient of $11.096 \pm 30$ mm. This value was obtained by taking the mean of calculated attenuation coefficients for $5$ runs at $300$s of lead-shielded Co-60 at different thicknesses. However, this calculated value carried a relative uncertainty greater than one. Therefore, we decided that it was insignificant in our analysis.
\par Although radiation count detection carries random uncertainty with it, this random uncertainty was taken into account through use of calculating the true count rate by a second-order Taylor Series approximation. Along with this random source of uncertainty, there were two systematic sources of uncertainty. The first source of systematic uncertainty came from our use of the two-source method to determine the dead time of our GM counter. In doing so, the respective sources were on separate shelves, one on top of the other, in our GM counter. The top source and shelf may have shielded some radiation from the bottom source, effectively lowering the measured count rate from the two sources from their true count rate. The other systematic source of uncertainty came from the assuming that the background count rate was the same with or without lead shielding. In our experiment, we measured the background count rate without lead shielding, effectively obtaining a higher background count rate than the true background count rate for the lead-shielded runs. 