\par The harmonic oscillator is a system that becomes displaced from its equilibrium position and experiences a restoring force proportional to its displacement. This system can be seen widely throughout the natural world in such examples pendulums, acoustics, masses attached to springs, and any small vibrational systems.\cite{Blinder} The forced, damped harmonic oscillator is a system that has a force driving the oscillations and a dissipative force acting on the object preventing some motion. The damping force is found in almost all physical systems as forces such as air-resistance, viscosity, and friction can all be found in systems that are not idealized under laboratory settings such as in a vacuum. 

%I dont know if this paragraph is necessary but it fills the page :/
\par Mathematically, the harmonic oscillator is a solution to a second order differential equation. Therefore we can describe the motion of an oscillating object with a mathematical equation. The solutions to these equations create three cases classifying the damping of the system. The first case, overdamped, is a system where the square of the damping factor is greater than four times the mass multiplied by the spring constant of the system and returns the system to equilibrium in a short period. The second, underdamped, is a system where the square of the damping factor is less than four times the mass multiplied by the spring constant and the system oscillates until it returns to equilibrium. The third and final case is the critically damped case where the square of the damping factor is equal to four times the mass multiplied by the spring constant. This case is a unique solution that returns the system to equilibrium the fastest.\cite{Meth}

\par A forced, damped harmonic oscillator has two separate solutions. The \textit{transient} solution is the solution to the damped oscillator. As time increases, this term approaches zero. The second solution is the \textit{steady-state} solution which is produced from the force applied on the harmonic oscillating system. This oscillation is similar to the oscillation of the natural frequency for free oscillations. The system reaches resonance frequency when the amplitude is the greatest in oscillation. In fact, we can compare the amplitude at the resonance frequency of this steady-state solution to the amplitude at the natural frequency of an unforced, undamped system with the same mass and spring constant. 

\par Comparing amplitude resonance is vital in understanding various physical situations for real world applications. Engineers apply this method to buildings that shift and sway in the wind. They compare the resonant frequency of the building to the natural frequeny, the natural unforced vibration of the object, and design the building to not have resonance when wind is incident on the building. This means that the building does not reach maximum amplitude while swaying due to the wind.

\par In this experiment, we compared the resonance frequency of this steady-state solution in a spring mass system as damping approaches zero to the natural frequency and compared the amplitudes of the oscillations at these frequencies. The amplitudes of a coupled two-mass system were also compared at these two frequencies.