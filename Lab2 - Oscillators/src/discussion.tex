\par In all three experiments, the amplitude resonance frequencies measured were statistically different from the theoretical values. This difference could have been caused due to the relatively large step size taken in varying the driving force. For example, analyzing Table \ref{tab::oscillationDouble} shows that the next recorded oscillation frequency after our measured $\omega_2 = 1.130 \pm 0.0005$ Hz was $1.217 \pm 0.0005$ Hz. It is entirely possible that a larger amplitude may have been measured if the driving frequency had been set closer to our theoretical expectation of $1.134 \pm 0.003$ Hz; however, since this value was skipped over, it is impossible to know without further experimentation. Additionally, since all of our measured frequencies were less than their theoretical counterparts, this would suggest that a damping factor was still observed on the air track during the experiment. In all theoretical calculations, the increase of damping factor would reduce the natural frequency as can be seen in the derivation of the single cart resonance frequency in equation (\ref{eq:naturalfreq}).

\par This damping could possibly have been caused by some friction on the air track still present in the experiment. The experiment where we added the weight to the single air cart system showed the greatest deviation from theoretical value. For the track to account for this gain in mass, the air track would have to produce a stronger output of air; this was impossible due to the experiment being run on the highest setting for the pump throughout. Another possibility could be the presence of turbulence under the carts. Because the system is moving and changing the direction constantly, the air has difficulty escaping and may get trapped and cause small changes in the air cart as it oscillates on the track. 

\par Due to the possibility of damping in the system, the observed maximum amplitudes might not have been the theoretical maximum possible in the system. From this we could estimate the damping factor $\gamma$ by relating the difference of theoretical resonance frequency to the measured resonance frequency of the system. This would be slightly unreliable because of the possibility that the maximums arent truly being measured as suggested. Therefore multiple factors of uncertainty would render this estimate difficult.

\par Though the maximum amplitudes were not achieved, the coupled, damped harmonic oscillator still demonstrated both resonance frequencies. Another common reference of these resonance frequencies are the symmetric and the asymmetric resonance. This is due to the way the carts oscillate in reference to each other. The carts moved for one frequency back and forth together in the same direction (asymmetric), and then opposite direction for the other frequency (symmetric). This nomenclature makes sense with reference to a center axis between the two carts.