Modern medical imaging using X-rays poses many challenges to doctors concerning the risk of cancer and damage to DNA. By creating a computational model for these x-rays using ray-tracing, the imaging process can be optimized to reduce these health risks. In this study, a monoenergetic effective energy value for X-ray spectra was studied for varying materials, thicknesses, and positions using the NIST attenuation coefficients for future use in a ray-tracing model. When comparing to simulated effective energy curves generated with SpekPy \cite{SpekPy}, the experimental curves for the monoatomic materials demonstrated a similar, but more rapid, linear relationship, while the compounds demonstrated an anomalous, exponential relationship. Additionally, it was found that varying the thickness and position of materials resulted in unexpected discrepancies in effective energy.