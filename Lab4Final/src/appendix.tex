\begin{lstlisting}[language = python, caption=Intensity Extraction Code, floatplacement=H]
    def get_relative_intensity(path, background_intensity=0):
    """
    Calculates the relative intensity of a square in the center of an image.
    :param path: Path to image
    :param background_intensity: Intensity of background
    :return: Relative intensity
    """
    # load image
    img = cv2.imread(path, 0)

    # convert to float and normalize
    img = img.astype(np.float32)/255

    # get mean pixel value and std dev of square in center of image
    center = (int(img.shape[0]/2), int(img.shape[1]/2))
    length = 100
    square = img[int(center[0]-length/2):int(center[0]+length/2), 
    int(center[1]-length/2):int(center[1]+length/2)]

    mean, std = cv2.meanStdDev(square)

    # if intensity is 1 or 0, return False, so they are not used as data points
    if mean[0][0] >= 0.999 or mean[0][0] <= 0.001:
        return False

    # obtain relative intensity by normalizing by background intensity
    # (I_square = I_background * I_object -> I_object = I_square / I_background)
    mean[0][0] = mean[0][0] / background_intensity


    # convert image to rgb
    img = cv2.cvtColor(img, cv2.COLOR_GRAY2RGB)

    # draw red square on image and relative intensity and std on image
    cv2.rectangle(img, (int(center[0]-length/2), int(center[1]-length/2)),
     (int(center[0]+length/2), int(center[1]+length/2)), (0, 0, 255), 3)

    cv2.putText(img, f"Intensity: {mean[0][0]:.3f} +- {std[0][0]:.3f}", (int(center[0]-length/2),
                int(center[1]-length/2)-10), cv2.FONT_HERSHEY_SIMPLEX, 1, (0, 0, 255), 2)

    return ufloat(mean[0][0], std[0][0])
\end{lstlisting}