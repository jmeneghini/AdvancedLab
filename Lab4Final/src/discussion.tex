The discrepancies between $E_{\text{eff}}$ for the monoatomic materials and the compounds have been explored extensively with little resolution. Several different theories have been explored, including the thickness of the sample, possible code errors, and the normalization of the image.

Upon looking for differences between the compounds and monoatomic materials, it was observed that both the compounds had a significantly larger thickness than the monoatomics, as seen in Table (\ref{table::matTable}). It was then considered whether or not the increased height of the sample could lead to an increased probability of x-rays scattering off the object and onto the detector. This effect would result in a seemingly larger background and could lead to unexpected $E_{\text{eff}}$ values. Upon manual examination of the images, the background was affected; however, the difference in intensity was minor and could not reasonably result in the exponential increase that was measured, especially since the discrepancies are still present in images where the background is non-existent.

Furthermore, since weighted averages of monoatomic attenuation coefficients were taken in the data processing code to obtain the attenuation coefficients for the compounds, it was considered whether or not a possible error was present in the code that could lead to this exponential behavior. Upon careful examination of the code, no errors were found, and for extra reassurance, a few images of the compounds were manually inspected, and verified the processing code's results.

Additionally, it was found that \textit{Heine \& Thomas} (2008) \cite{Heine} performed a similar $E_{\text{eff}}$ calibration using a mammography system. While their overall technique was similar to the methodology of this experiment, one key difference was the normalization factor used in the processing of the intensity data. In particular, they normalized the pixel value by $mAs$, which is the tube current multiplied by the exposure time. This effectively would be the incident intensity of the x-rays and is equivalent to $I_0$. However, this normalization was not able to be done for this experiment, since the C-arm does not provide any data on the exposure time for images. Still, this normalization would be the same for each kVp; therefore, this cannot explain the drastic differences between the $E_{\text{eff}}$'s of the monoatomic and compound materials that were measured at higher kVps. On the other hand, since the current of the tube was provided for each image and increased at a rate proportional to the kVp, this could explain why the $E_{\text{eff}}$'s of the monoatomic materials increased at a rate greater than what was predicted by the simulation. 

Finally, combining these results with the unexpected differences in $E_{\text{eff}}$ that were measured in the offset and varied thickness experiments, it is possible the x-ray machine may be doing some sort of post-processing of the images that is not described in the manual \cite{CArm}. Without knowing the specifics of this processing, the original, unfiltered images can not be calculated. Overall, the measurements obtained from this experiment reveal that the $E_{\text{eff}}$ approach for ray-tracing cannot be used for the GE MiniView 6800 Mini C-Arm. As an alternative, research collaborator Dr.\ Gregory Bisignani has offered us his clinical X-ray machine for use in further experimentation. Likely, this offer will be accepted and the experiment described in this report will be repeated with his machine.

